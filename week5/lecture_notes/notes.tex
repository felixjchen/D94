\documentclass[12pt]{article}
\usepackage{enumitem}
\usepackage{amssymb}
\usepackage{amsmath}
\usepackage{amsfonts}

\usepackage{clrscode3e}

\usepackage[left=0.5in,right=1in]{geometry}
\renewcommand{\baselinestretch}{1.5}
\newcommand{\sigmastar}{\Sigma^*}


\begin{document}

\section{}

\subsection{Definition: Code}
A code $C$ of block length $n$ over a finite alphabet $\Sigma$ is any subset of $\Sigma^n$.

$$ C \subseteq \Sigma^n$$

\begin{center}
  "the set of all possible codewords"
\end{center}

e.g. $\Sigma = \{0,1\}$, $n=5$, $C = \{00000, 11111, 00001\}$

\subsection{Definition: Dimension of a Code}
Given a code $C \subseteq \Sigma^n$, $C$ has dimension $k$ defined by:

$$ k = log_{\lvert \Sigma \rvert} \lvert C \rvert $$

\begin{center}
  "n is the size of any codeword"
\end{center}
\begin{center}
  "k is the size of the decoded codeword"
\end{center}

Note: $k \leq n$

e.g. $\Sigma = \{0,1\}$, $n=5$, $C = \{00000, 00001, 00010, 00011, 00100, 00101, 00110, 00111\}$ then, $ k = log_2 (8) = 3$

\subsection{Definition: Rate of a code}
Given a code $C \subseteq \Sigma^n$ with dimension $k$, $C$ has rate $R$ defined by:
$$ R = \frac{k}{n}$$

\begin{center}
  "R is the ratio of non-redundent bits, higher is better, lower means more rendency"
\end{center}


\subsection{Definition: Hamming Distance}
The Hamming Distance between two equal length strings is the number of elementwise differences.

$$d_H = \lvert \{ i \mid x_i \neq y_i \} \rvert$$

e.g. $d_H(bbb, aaa) = 3$

e.g. $d_H(xyz, abc) = 3$

\subsection{Definition: Minimum distance of a code}
Given a code $C \subseteq \Sigma^n$, $C$'s minimum distance $d$ is the smallest distance between any two codewords in $C$.

$$d = min \{d_H(i,j) \mid i,j \in C, i \neq j \}$$


\break

    
\end{document}