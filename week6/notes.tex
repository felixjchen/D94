\documentclass[12pt]{article}
\usepackage{enumitem}
\usepackage{amssymb}
\usepackage{amsmath}
\usepackage{amsfonts}
\usepackage{mathtools}

\usepackage{clrscode3e}

\usepackage[left=0.5in,right=1in]{geometry}
\renewcommand{\baselinestretch}{1.5}
\newcommand{\sigmastar}{\Sigma^*}

\DeclarePairedDelimiter\floor{\lfloor}{\rfloor}

\begin{document}

\section{Generator and Parity Check Matricies}

\subsection{Recall: Linear Code}
We say that $C \subseteq \Sigma^n$ is a linear code if $C$  is a linear subspace of $\Sigma^n$ where $\Sigma$ is a finite field. i.e.:

\begin{enumerate}
  \item $0 \in C$
  \item $\forall a,b \in C, a+b \in C$ 
\end{enumerate}

\begin{center}
  "a linear code is an error-correcting code for which any linear combination of codewords is also a codeword"
\end{center}


\subsection{Definition: Generator Matrix}
Given some linear code $C \subseteq \Sigma^n$, we can create a basis that spans $C$. Let $G$ be a generator matrix, who's rows form a basis for $C$. We can use $G \in \mathbb{R}^{k \times n}$ to generate codewords given a message $m \in \Sigma^k$:


  $$\underbrace{c}_\text{$1 \times n$} = 
  \underbrace{m}_\text{$1 \times k$} \underbrace{G}_\text{$k \times n$} $$


Where $c$ is some codeword.


\subsection{Definition: Parity Check Matrix}
Given some linear code $C \subseteq \Sigma^n$, a parity check matrix $H$ can be used to check if a codeword $c \in C$.

$$ \underbrace{H}_\text{$(n-k) \times n$} \underbrace{c^T}_\text{$n \times 1$}  = \mathbf{0} \iff c \in C  $$

\section{Hamming Codes}

\begin{center}
  "Hamming was interested in two problems at once: increasing the distance as much as possible, while at the same time increasing the code rate as much as possible."
\end{center}

\subsection{Definition: Parity Bit}
A parity bit is a bit, it's added to a string of bits to ensure the total number of 1's in a string is even or odd. 

e.g. Even parity bit is added to make the total number of 1's even. 

\begin{center}
  \begin{tabular}{ c | c | c }
   string & number of 1's & even parity bit \\ 
   \hline
   0001000 & 1 & 1 \\  
   0001111 & 4 & 0 \\
   0101010 & 3 & 1     
  \end{tabular}
\end{center}

\subsection{Definition: Hamming Code}
    
\end{document}